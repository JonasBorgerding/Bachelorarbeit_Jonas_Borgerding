\documentclass{article}
\usepackage{ngerman}
% Harte und weiche Kriterien formulieren für Sprachen, Webframeworks
% Harte Kriterien bei Sprache: Unter top 20 im TIOBE Index Oktober 2019-2020 Hochsprache? Geeignet für Webentwicklung?
% Weiche Kriterien bei Sprache: Performance (wiedersprüchlichkeit raussuchen und selbst messen); Entwicklungsgeschwindigkeit
% versuche präzise zu formulieren

% IEEE Web Framework sprachen
% https://spectrum.ieee.org/ns/IEEE_TPL_2019/index/2019/1/0/0/0/1/50/1/50/1/50/1/30/1/30/1/20/1/20/1/5/1/50/1/100/1/50/
% Mögliche Quellen:
% https://thenewstack.io/which-programming-languages-use-the-least-electricity/
% https://dzone.com/articles/top-programming-languages-rankings
% Diese Quelle enthält Widersprüche: https://attractivechaos.github.io/plb/
\begin{document}%XHQ Siemens
    \pagestyle{empty}
    \noindent
    Web Entwicklung ist von Siemens Seite aus möglich, da keine direkten Funktionalitäten bekannt für Neuentwicklung\\\\
    \textbf{Kriterien:}\\
    \textit{Technologieplattformauswahl:}\\
    Folgende Technologien werden betrachtet: Native Entwicklung, Hybride Entwicklung, Cross-Platform Entwicklung und Web Entwicklung.\\
    %Folgende Quellen werden verwendet: Heise.de, academic.microsoft.com, scholar.google.com, google.com
    Quellen werden über Google mit den Suchanfragen nach "{}Web Development"{}, "{}Native Development"{}, "{}Hybrid Development"{} und "{}Cross-Platform Development"{} gesucht.
    \begin{enumerate}
        \item Verfügbarkeit auf allen Betriebssystemen, die eine Benutzeroberfläche und einen Browser haben(Windows, iOS, Android, Windows Phone, Blackberry etc.)
        \item Keine App-Store-Abhängigkeiten, da in der Regel kein Internet auf den Endgeräten verfügbar ist
        \item schnelle und kostengünstige Wartung und Pflege, d.h. wenig Zeit notwendig um Updates zu entwickeln und bei dem Kunden vor Ort installieren.
    \end{enumerate}
    \textit{Sprachauswahl:}\\
    Ab hier wird die Technologieplattform Web-Entwicklung betrachtet.\\
    Folgende Harte(Binäre) Kriterien werden für die grobe Sprachauswahl verwendet:
    Hier wird in Suchmaschinen nach "{}TIOBE Index"{} gesucht, es wird die Time Maschine für Webseiten verwendet, um den TIOBE Index der letzten Jahre zu sehen. Weiterhint wird nach "{}Sprache Web Framework"{} gesucht, um herauszufinden, ob für diese Sprache ein Web-Framework existiert.
    \begin{enumerate}
        \item Sprache ist unter top 20 der beliebtesten Sprachen aus dem TIOBE Indexvon Oktober 2019 bis Oktober 2020
        \item Sprache ist eine "{}Hochsprache"{}(z.B. C, aber nicht Assembly)
        \item Sprache ist für die Web-Entwicklung geeignet. Mindestens 1 Framework oder eine Erweiterung hat einen GUI Designer.
    \end{enumerate}
    Folgende Weiche Kriterien werden für die feine Sprachauswahl verwendet:
    Als Quelle dient die Suche nach "{}Sprache execution performance comparison"{}, "{}programming language memory usage comparison"{}
    \begin{enumerate}
        \item Execution Performance. Hier werden Performance-Vergleiche herausgesucht(programming languages performance comparison). Von diesen werden die Sprachen genommen, die die 5 besten Ergebnisse liefern. Falls es nicht eindeutig ist, werde ich selbst kleine Testprogramme schreiben(for loop bis 1000000 1. ohne berechnung, 2. inkrementierung einer Variable um 1 und 3. Konkatenation von einem String. Diese 3 Tests werden jeweils für die Sprachen 10 mal durchgeführt und der Mittelwert übernommen).
        \item Development Performance. Hier wird versucht herauszufinden, wie schnell eine Entwicklung in einer Sprache im Vergleich zu einer anderen Sprache geht. Hierzu werden Quellen herangezogen, die angeben, wie viele Bibliotheken die Sprache bereitstellt.
        \item Memory Usage. Hier werden Speicherplatz- und Energieverwendungsvergleiche für die Sprachen herausgesucht(programming language memory uasage comparison).
        \item Zukunftssicherheit. Hier wird die Sprache hinsichtlich der Zukunftssicherheit untersucht. Dabei wird geschaut, ob die Sprache in den letzten 18 Monaten ein Update erhalten hat und wie sich der Verlauf der Sprache in Hinsicht der Beliebtheit in den letzten 5 Jahren entwickelt hat. Anschließend wird eine Progrnose abgegeben.
    \end{enumerate}
    \textit{Frameworkauswahl:}\\
    Für die Frameworkauswahl werden folgende harten Kriterien verwendet:
    Gesucht wird in Suchmaschinen nach "{}Sprache(z.b. Java) Web Framework"{}, auch in Kombination mit "{}Designer"{}, "{}Databinding"{} und "{}Drag and Drop"{}.
    \begin{enumerate}
        \item Frameworksprache ist unter den resultierenden Sprachen aus vorherigem Schritt
        \item Framework unterstützt das template system
        \item Framework unterstützt Databinding
    \end{enumerate}
    Für die feine Frameworkauswahl werden folgende weiche Kriterien verwendet
    \begin{enumerate}
        \item Zukunftssicherheit. Hier wird untersucht, wer dieses Framework entwickelt hat(steckt ein großes Unternehmen oder eine Open-Source Community dahinter?) und ob dieses Framework in den letzten 12 Monaten ein Update erhalten hat.
        \item Development Performance. Hier wird untersucht, wie viel Aufwand und Code notwendig ist um eine neue Seite ohne Inhalt in dem Framework zu erstellen. Je weniger selbst geschrieben werden muss, desto besser. Hier werden maximal 3 Frameworks anhand des Aufwands zur Erstellung einer Seite und der Zukunftssicherheit ausgewählt.
    \end{enumerate}
\end{document}

%performance
%zukun